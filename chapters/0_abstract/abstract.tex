\chapter*{Abstract}

The presented work describes the extension of the cids navigator, the default user interface of the cids platform, with the ability to allow an integration of web applications as cids renderer and editors. 
Cids renderer and editors are GUI components that provide the user with relevant information for a special object of interest. 
With the integration of web applications that can cover the same functionality, totally new possibilities in designing systems based on the cids platform are reached. 
Renderer and editor components based on web technologies offer the chance to reuse them for developing web applications that can profit from cids platform features. 
Hence the cids platform is totally Java based, it is not possible to supplement cids based systems with web applications. 
Web renderer and editor components are one necessary step to overcome this limitation.

The outlined approach for integrating web applications is highly flexible. 
No assumption on the type of the web application that can be integrated is made. 
Just the degree how seamless the integration of the web application is shaped, depends on the type of the web application. 
The presented solution in this respect is very scalable and can reach from the simple display of web pages up to highly integrated renderer and editor components, that are embedded in a seamless manner. 
Especially the latter case is proved with a custom developed web application that replaces the functionality of an already existing renderer and editor component for survey plans. 

Furthermore this work provides a examination of the nowadays existing web technologies which are analysed in regard for developing web renderer components described above. The focus is put on modern JavaScript MVC frameworks, for whose an extensive and detailed analysis is given and a framework for web development projects is examined.

