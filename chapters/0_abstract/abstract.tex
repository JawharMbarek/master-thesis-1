\chapter*{Abstract}

The presented work describes the extension of the cids navigator, the default user interface of the cids platform, with the ability to allow an integration of web applications as cids renderer and editors. 
Cids renderer and editors are GUI components that provide the user with relevant information for a specific object of interest. 
With the integration of web applications that can cover the same functionality, totally new ways in designing systems based on the cids platform are possible. 
Renderer and editor components based on web technologies (cids web renderer and editor components) offer the chance to reuse them when developing web applications that make use of cids platform features. 
Hence the cids platform is totally Java based, it is not possible to combine cids based systems with web applications. 
Cids Web renderer and editor components are one necessary step to overcome this limitation.

The outlined approach for integrating web applications is highly flexible. 
No assumption on the type of the web application that can be integrated is made. 
Just the degree how seamless the integration of a web application as cids renderer or editor can be realized, depends on the type of the web application. 
The presented solution in this respect is very scalable and can reach from the simple display of web pages up to highly integrated renderer and editor components, that are embedded in a seamless manner. 
Especially the latter case is proved with a custom developed web application that replaces the functionality of an already existing cids renderer and editor for survey plans.
Furthermore this work provides an examination of the currently existing web technologies which are analysed with regard for developing web renderer an editor components as described above. 
The focus is put on modern JavaScript MVC frameworks, for whose an extensive and detailed analysis is given. 
The result of this analysis is a candidate framework that facilitates the developments of web renderer and editor components .

